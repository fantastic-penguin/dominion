\documentstyle[12pt,latexinfo,tabular]{report}
\pagestyle{headings}

\begin{document}

\newindex{cp}
\newindex{vr}
\newindex{fn}
\newindex{tp}
\newindex{pg}
\newindex{ky}

\comment %**start of header (This is for running Texinfo on a region.)
\setfilename{dominion.info}
\markboth{Dominion Manual}{Dominion Manual}
\comment %**end of header (This is for running Texinfo on a region.)

\begin{tex}
\pagestyle{empty}
\title{Dominion Manual}

\author{The Dominion Project}

\date{\today}

\maketitle
\pagestyle{headings}
\pagenumbering{roman}
\tableofcontents

\comment  The following two commands start the copyright page.
\clearpage
\comment \vskip 0pt plus 1filll

Copyright \copyright{} 1990 Free Software Foundation, Inc.

Permission is granted to make and distribute verbatim copies of this manual
provided the copyright notice and this permission notice are preserved on
all copies.

Permission is granted to copy and distribute modified versions of this
manual under the conditions for verbatim copying, provided also that the
GNU Copyright statement is available to the distributee, and provided that
the entire resulting derived work is distributed under the terms of a
permission notice identical to this one.

Permission is granted to copy and distribute translations of this manual
into another language, under the above conditions for modified versions.

\clearpage
\end{tex}

\begin{ifinfo}
This file documents dominion.

Copyright @copyright{} 1990 Free Software Foundation, Inc.
Authored by the dominion project (see Authors section of this manual).

Permission is granted to make and distribute verbatim copies of this manual
provided the copyright notice and this permission notice are preserved on
all copies.
\begin{ignore}
Permission is granted to process this file through Tex and print the
results, provided the printed document carries copying permission notice
identical to this one except for the removal of this paragraph (this
paragraph not being relevant to the printed manual).
\end{ignore}
Permission is granted to copy and distribute modified versions of this
manual under the conditions for verbatim copying, provided also that the
GNU Copyright statement is available to the distributee, and provided that
the entire resulting derived work is distributed under the terms of a
permission notice identical to this one.

Permission is granted to copy and distribute translations of this manual
into another language, under the above conditions for modified versions.
\end{ifinfo}

\clearpage
\pagenumbering{arabic}

\chapter{Overview}

\strong{Dominion} is an empire-style multi-player world simulation game.
Each player is the leader of a nation, and makes decisions for that
nation.  The decisions are political, military, diplomatic and
economic, and all these are extremely important for the well-being of
a nation.  Some nations can be played by the computer.  These nations
are called NPCs (Non Player Countries).  They play a challenging game,
and are quite useful if few human players are available.

Dominion has features from both fantasy role-playing games, educational
games, and war games: a user needs to develop a character as leader of
a nation, keep a healty economy, and can then develop a strong
military force using magic or technology.

Most of the moves you make are not resolved until the end of a turn,
when the \emph{update} is run.  This update will incorporate your
changes into the world data base, then it will update your economy,
handle migration of people, resolve battles and conquest of land,
freshen up your armies (restore move points), and a few other things.

The rhythm of the game is set by how much money you have spent, and
how much you have moved your armies.  If you have spent all your
money, you will need to wait until your revenue comes in (after the
update) before you can spend more.  Similarly, if you have already
moved all your armies, and they are overcome by fatigue, then you will
have to wait until they have recovered.  This will happen in the
update.  The time elapsed between updates is called a \emph{thon}.

\chapter{Getting started}

Your nation is added to the game by the Game Master, or someone
trusted by the Game Master.  You get to choose the \emph{name},
\emph{leader name}, \emph{race}, \emph{nation mark} (used for display)
and \emph{magical order} for your nation.  The choice of race and
magical order are important: each race comes with certain
characteristics (dwarves are better miners; elves are more
intelligent, have good magical aptitude; orcs reproduce like crazy and
so on\dots{}).  The magical order determines what spirits you will be
able to summon, and which spells will be available for you to cast.
Apart from the advantages of each race and magic order, they largely
determine what role you will play in the game.

When you first look at your country you will see sectors with letters on
them.  Each of these is a sector designation from which you can tell
what your country uses the sector for.  You can change these
designations once you own a sector; this will typically cost money,
and maybe metal and jewels.

\section{Moving around the map}
You can move to another sector on the map using the [h], [j], [k], [l]
keys to move up, down, right, left; and the keys [y], [u], [b], [n] to
move diagonally.  This is similar to the cursor movements in some
editors (like \emph{vi}), and some UNIX games (such as \emph{rogue},
\emph{larn}, \emph{nethack}, \emph{conquer}, \dots{}).

Alternatively, you may use the numeric keypad, in which case the
number keys represent the same direction in which they point on the
keypad.  Both the ordinary keys and the numeric keypad can be used to
browse the map and to move armies.  The following diagram shows you
the directions and keys you can use.

For large movements across the map, you can use the \emph{upper-case}
letters [H], [J], [K], [L].  These jump 8 sectors over.

One last thing: when you are starting out in the game, you might be
overwhelmed by the number of commands, options, report screens, and so
on\dots{}.  You might also be overwhelmed by the size of this manual,
and not want to read it all until you have been playing a few turns.
The way out is to just explore the commands described in the reference
card.  Except at main level of the game (where you look at your
nation's map), all the commands have menus for the subcommands, and
you can usually follow those.  Moreover, we have a context-sensitive
help system, so at any moment you can type [?] and get that part of
the manual which regards the command you are currently using.

\comment hmm, think about keeping this on the same page.
\comment \newpage
\begin{verbatim}
                               NORTH

                                (K)
                          y,7   k,8   u,9
                             \   |   /
                               \ | /
          WEST      (H) h,4 ---- 0 ---- l,6 (L)       EAST
                               / | \
                             /   |   \
                          b,1   j,2   n,3
                                (J)

                               SOUTH
\end{verbatim}

\section{Moving your armies}
Armies and navies are manipulated with the [a] menu.  Here are
\emph{some} of the army commands.  A complete list comes later.

\begin{itemize}

\item
\emph{[l]ist} army types will give you a list of army and navy types
available to you for drafting.  As your technology improves, more army
and navy types will become available.  At the start, you should only
see \emph{Cavemen} and \emph{Caravans} in the list.

\item
\emph{[n]ext army}
\item
\emph{[p]revious army}.  These pick the next and previous armies in the
current sector, skipping armies that do not belong to you.  The
currently ``picked'' army should be highlighted.

\item
\emph{[N]} and \emph{[P]} are like [n] and [p], but they move the the next
army \emph{number}, whether it is on the same sector or not.  These
can be used to cycle through all the armies in your nation.

\item
\emph{[m]ove}. after you have selected an army you want to move just type
[m] and move the cursor to where you want the army to be.
\emph{Warning}: watch your army's \strong{move points} and the
\strong{move cost} of each sector you cross: if you run out your army
might get stuck where you don't want it!!

\item
\emph{[s]tatus}.  This lets you change your army's status.  The status is
the mode your army is in, and affects what your army will do in
various situation.  The most basic army statuses are:

\begin{itemize}
\item
\emph{[a]ttack}. When in this mode your army will attack another
army in the same sector.  This will happen only if you are at WAR
or JIHAD.

\item
\emph{[d]efend}. An army in defend status can not take a sector nor
will it attack another army.  If attacked it will defend itself.

\item
\emph{[o]ccupy}. If you want to take a sector that is un-owned or owned
by an enemy, you move an army to that sector and set its status to
occupy.
\end{itemize}

\item
\emph{[d]raft}.  To draft an army or navy you must be in a city or
capital that you own.  Drafting an army will cost metal and money, so
be careful and watch how much you spend.  You will be given the list
of available army types, (see [l]ist), and you must just type the
abbreviation character for that army type.

\end{itemize}

\section{Taking sectors}
At the beginning of the game, you should take many sectors: this gives
a safety buffer around your capital, and allows you to look for
resources in the new occupied lands.  You take sectors by moving an
army \emph{of at least 100 soldiers} to that sector and setting it on
occupy mode.  If the sector belongs to another nation, you will have
to declare war on them to take the sector from them.  You declare war
using the [r]eports menu, and choosing [d]iplomacy.

Once you take a sector (it will become yours after the update), you
can redesignate that sector so it produces what you want.  For
example, redesignating to a farm will make that sector produce food,
and so on.  Sector designations are described in great detail later,
but you should know now that to change a designation you use the
[Z]oom key to focus on the specific sector, and then you can change
the designation with the [r]edesignate key \emph{inside} the [Z]oom
screen.  You will be given a menu of possible designations.

To help you choose a designation for the sectors you take, the sector
window (to the bottom right of your screen) shows you the \strong{soil},
\strong{metal}, \strong{jewels} in that sector.  If the soil value is
high, then redesignating to a \strong{farm} is a good idea.  If the
sector has a high metal yield, you might want to make a \strong{metal mine}
out of it, and so on.

Below is an example of a sector window that shows a sector with jewels
5, metal 0 and soil 6.  The sector belongs to nation Khazad Dum, has
coordinates (2, 2) relative to the current player, has 452 inhabitants
which are of race (D) (Dwarves).

\comment Hmm:  think about putting it on the same page or a new page
\comment \newpage
\begin{verbatim}
                                +----------------------+
                                |(2,2)                 |
                                |Khazad Dum-jwl. mine  |
                                |Brush Plateau         |
                                |452 people (D)        |
                                |metal 0     jewels 5  |
                                | soil 6   movecost 1  |
                                +----------------------+
\end{verbatim}

\section{Setting up your budget}
It is important that you set up your budget properly.  You do
this by hitting [r] ([r]eport), and then [b] (for the [b]udget
report).  In here you should choose your tax rate, and the amount of
money/metal/jewels you choose to invest in various types of research.
A note about taxes: if your taxes are too high, your production of
food, metal and jewels will decrease.

There are default values set for investment in magic and technology,
and there is a default tax rate.  This is intended to guide new
players through their first moves.

In the budget report you can also spend a fraction of your
\emph{reserves} on various types of research.  To do this
you use the [s]torage option, and spend what you want of your current
treasury.  This amount will be reset to 0 after each update.

\section{Checking out your info screen}
You should check your info screen every thon.  This shows parameters
of your race and nation, including your skills in magic and
technology, combat, espionage and so forth.

\section{Commands (brief)}
This is a brief list of dominion commands.  It is actually a verbatim copy
of the reference card available as on-line help in dominion (you access
this with the [?] key followed by [r]).
\newpage
\begin{verbatim}
                       Dominion QUICK REFERENCE CARD
Movement (you are where the `*' is):
                                   NORTH
                                    (K)
                              y,7   k,8   u,9
                                 \   |   /
                                   \ | /
              WEST      (H) h,4 ---- * ---- l,6 (L)       EAST
                                   / | \
                                 /   |   \
                              b,1   j,2   n,3
                                    (J)
                                   SOUTH
Display:
    [d]isplay options     [F] dump map to file   [w]indow manipulation
    [^L] redraw screen    [p] jump to a point    [P] jump to your capital
Administration:
    [r]eports             [a]rmies               [Z]oom on sector
    [W]izardry            [t]ransportation       [C]onstruct
Miscellaneous:
    [Q]uit (or [q]uit)    [m]ail                 [N]ews
    [O]ptions

\end{verbatim}

\chapter{What is in the world}

Here we describe races, terrain, technology, magic, designations,
communications\dots{}

\section{Races}
There are several races in dominion, and each Game Master can add any
races s/he pleases by modifying the race descriptor file.

The parameters describing your race are strength, reproduction,
mortality rate, intelligence, speed, stealth, the race's preferred
altitude, vegetation, and temperature, aptitude for magic, farming and
mining.

In addition to these parameters, each race has certain special army
types available to it.  For example, Orcs can draft armies of type
``Orc'', which have some advantages, and Harpies can draft armies of
type ``Harpies'' which can fly.  Ogres have the army type ``Ogres'',
which fight very well; and Hobbits can draft ``Hobbits'', who are
hidden from their enemies' sight.

The races currently available are Elf, Human, Dwarf, Orc, Merfolk,
Icefolk, Hobbit, Gnome, Harpy, Ogre, Walrus, Algae, and Squid.  If you have a
race you would like to play and do not see it listed, talk to your
game administrator about adding it to the game.  This should not be
done lightly, because the parameters describing each new race have to
be tuned to preserve game balance.

Some of the races (Merfolk, Walrus, Algae, and Squid) live under water.  The
game is almost symmetrical for races that live above and below water,
though there are some differences.  Races that live above and below
water can interact (and fight) in several ways, as described below.

Each of the above listed traits effects the races in the following ways:

\begin{itemize}
\item
Strength: affects your combat bonus.
\item
Reproduction: the rate at which people are born in your nation.
\item
Mortality: every year this percentage of your population dies.
\item
Intelligence: affects your acquisition of technological skill, and
also helps in combat.
\item
Speed: affects your armies' move points.
\item
Stealth: affects your spy and secrecy skills.
\item
Preferred altitude, preferred vegetation, preferred temperature: these
affect migration of people, and move cost too.  If your race likes high
temperatures, deserts will have less move cost.  If they like high
altitude, mountains will have less move cost.
\item
Magic aptitude: this affects how quickly you learn the spells for the
nation's chosen magical order, and how many spell points you gain.
\item
Farming: your farms produce this percentage more than their basic
productivity.  This will increase during the game, as you acquire new
technology.
\item
Mining: your metal and jewel mines produce this percentage more than
their basic productivity.  This will increase during the game as you
acquire new technology.
\end{itemize}

\subsection{Basic races shipped with dominion}
The Race descriptor table, below, shows the parameters for each race.

\begin{same}
\begin{table}[hbpt]
\caption{Race Types}
\begin{tabular}{ || l | r | r | r | r | r | r | r | r | r | r | r | r || }
\hline
Race   &Str&Rep&Mort&Intel&Spd&Stl&Alt&Veg&Temp&Mag&Farm&Mine\\
\hline
Master  & 0& 0& 0& 0&  0&0&0& 0& 0& 0& 0& 0\\
Human   &80&11& 8&50& 65&4&2& 3& 7&30& 0& 0\\
Elf     &70& 8& 5&70& 80&8&2& 5& 7&55&10&-15\\
Orc     &50&15&10&20& 40&3&4& 4& 4&35& 0& 5\\
Dwarf   &95& 9& 6&60& 40&2&5& 3& 6&30&-5&20\\
Hobbit  &15&10& 7&45& 50&9&3& 4& 7&20& 5& 5\\
Merfolk &30& 7& 4&75& 80&7&-2&-1& 4&55&50&-10\\
Icefolk &90& 9& 7&50& 70&4&3& 0&10&30&50& 0\\
Gnome   &75&10& 8&95& 40&9&4& 3& 5&10& 0&10\\
Harpy   &40&12&10&25& 60&5&5& 4& 6&30& 0&-5\\
Ogre    &95& 5& 3&75& 60&5&4& 3& 6&50& 5& 5\\
Walrus  &95& 9& 7&30& 80&3&-1&-1& 7&55&50& 5\\
Algae   &15&11& 6&50& 60&7&-2&-1& 3&45&80& 0\\
Squid   &40& 9& 8&25& 110&8&-2&-1& 5&40&50&15\\
\hline
\end{tabular}
\end{table}
\end{same}

\subsection{A brief history of each race}
\indent \strong{Elves} are a race of great wisdom and magical skill.  They do
not die unless killed in battle.

\strong{Humans} are stronger than elves, but less wise, less magical, and
they have shorter life spans.

\strong{Orcs} are an evil mockery of elves.  They come from elves that were
captured by the Dark Lord and twisted into horrible shapes.  They
reproduce very quickly, but are stupid, slow and weak.

\strong{Dwarves} are sons of the earth.  They are very strong, as they were
made to withstand the world in an evil age.  They love mountains and caves,
and typically build their kingdoms in great networks of caves.

\strong{Hobbits} are merry little folk.  They meddle little into the affairs
of men and elves, and are not a warlike people.  They can move very quietly.

\strong{Merfolk} are people who live in the water.  They are strong in magic.

\strong{Icefolk} are creatures of the polar caps and glaciers.  They like
the cold temperatures.

\strong{Gnomes} are very technologically apt, but deal little with magic.

\strong{Harpies} are stupid and vicious.  Their only strength is that they can
draft armies of flying Harpies.

\strong{Ogres} are slow but strong.  They are quick to anger but soon forget
what they were angry about.  They do not reproduce much, but are extremely
inteligent.

\strong{Walrus} are water creatures with a preference towards colder climates.
They would rather sit on an iceberg than a warm sunny beach.

\strong{Algae} are water creatures that move little, are quite weak against
foes, but handy with magic and farming.

\strong{Squid} are very fast water creatures, with decent aptitudes.

\subsection{Land-water interaction}
For the most part, land races operate on land and water races in the
water.  But each can extend its influence to the other side of the sea
level in various ways.

Ships, when owned by a land race, will travel on water.  The converse
is true for a water race: its ships will travel on land.

If a land nation sees some good mines or farms in the water, it can
occupy that sector using {\em Swimmers} or {\em Scuba_divers} army
units which have the \key{W} (Water) flag.  Similarly, a water race
can occupy land sectors using armies such as {\em Walkers} which have
the \key{L} (Land) flag.  Notice that any other army or spirit with
the W and L flags will work fine.

The army types mentioned above all have the \key{f} (front line) flag,
so they can be unloaded from ships onto the desired sector.

Once the sector is occupied you cannot move people into it or they
will drown (or suffocate).  But you can (at great expense) build a
bubble over the sector (or cast a change-altitude spell) which will
allow your people to move into it, even if it is a water (or land)
sector.  This provides a means for colonizing the oceans (continents).

\section{Technology}
A nation starts with very low skills in mining, farming, construction,
etc\dots{} These skills can be developed by investing money or metal
in technology research.  This investment is made in the budget report.
Increase in technology depends on how much metal and money you invest
each thon.  Metal will increase technology proportional to the
\emph{4/3's root} of the amount of metal invested;
money will increase technology proportional
to the \emph{square root} of the amount of metal invested

Other benefits of technological R\&D are that with each new technology
you develop, you can gain certain things.  For example, \strong{fire}
increases mining ability, decreases the mortality rate, and increases
farming ability.  As well, you can gain the ability to draft
new types of armies.

\section{Magic}
Your nation is initiated to one of the magic orders available.

Each order is characterized by the set of \strong{spells} known to that
order, and a set of \strong{spirits} that can be summoned by mages of
that order.  When you are initiated as a national leader you know only
a little of the magic of that order, but you can increase your
knowledge by investing money and jewels into magical research.  As you
invest more and more, your magic skill will increase, and you will
learn more advanced spells, and how to summon more powerful spirits.
This investment is made with the budget report.

As with technology, your magic skill increase is proportional to the
jewels invested, and to the square root of the money you invest.

On top of this, some magical orders bring a set of characteristics to
its initiates.  These are described order by order.

To \emph{use} magic, i.e. to cast spells and summon spirits, you must
acquire \strong{spell points}, and initiate one or more mages.  You get
spell points in proportion to the amount of \emph{jewels} you invest
in magical R\&D, but \emph{not} from your money investment.

You need to bring a mage to a certain sector to cast a spell in it or
summon a spirit there, so you should make sure you initiate some mages
to work for your nation.  They cost a lot to initiate and maintain,
but are worthwhile.  Mages are initiated in your temples and
universities, which are located in either the city or the country.

The [W]izardry command allows you [l]ist available spells and spirits,
[c]ast spells, [s]ummon spirits, and [i]nitiate mages.

The basic magic orders in dominion are:

\begin{itemize}

\item
Order of \emph{Aule}.  Aule is the god of the earth, and is interested
in all that happens in the depths of the earth.  He protects miners
and workers of metal and stone, and gives them an extra 10% mining
skill.

\item
Order of \emph{Avian}.  This order allows you to summon spirits 
related to the air.

\item
Order of \emph{Chess}.  This order allows you to summon spirits
similar to those on the chessboard in their movements and strength.

\item
Order of \emph{Demonology}, concerned with the conjuring of demons.

\item
Order of \emph{Diana}, mostly concerned with animals and hunting.

\item
Order of \emph{Inferno}.  This order is concerned with power through
fire.

\item
Order of \emph{Monsters}, this order has available to it many
monstrous creatures.

\item
Order of \emph{Necromancy}, concerned with the invocation of dead and
undead spirits.  Nations of this order start with their death rate
increased by 2%.

\item
Order of \emph{Neptune}.  Neptune is the god of the oceans, and his
order is concerned with the waters.

\item
Order of \emph{Time}.  This order steps back to the age of the
dinosaurs.

\item
Order of \emph{Unity}.  This order has spirits that combine different
creatures in one body.

\item
Order of \emph{Yavanna}.  Yavanna is the godess of plants, and
basically everything that grows and is fertile is under her
protection.  Yavanna gives her initiates an extra 50% farming skill.

\item
Order of \emph{Insects}.  This order has spirits from the insect
world.  This initiation gives a nation 1% extra reproduction.

\end{itemize}

The Game Master can add other magical orders to the game by inserting
a list of spirits for that order into a file.

\section{Designations}
If you own a sector, you can \emph{redesignate} it.  That is, specify
what function that sector has.  It costs a certain amount of money
(and possibly also metal, jewels, \dots{}) to redesignate a sector.

Each type of sector can employ a different number of people.  For
example, a city can employ several thousand people, whereas a farm
sector can only properly employ a few hundred people.  The \emph{basic
number} of people that a sector can employ is listed in the table
below.  This value is then modified by how much your race tends to
crowd.  If you are an orc, for example, more people can be crammed
into a single sector, so the formula is:
\begin{ifinfo}
	max_employed = sector_max * sqrt(reproduction / 10).
\end{ifinfo}
\begin{tex}
\[ max\_employed = sector\_max \times \sqrt{\frac{reproduction}{10}}. \]
\end{tex}

The possible designations (together with the characters that are
displayed on the map) are:
\begin{itemize}
\item
Farm (\kbd{f}) - these produce food.  Without farmers producing food,
your people will starve to death.  Farms also generate revenue in the
form of taxes.
\item
Metal mine (\kbd{m}) - supplies your country with metal, proportional to
how good the metal mine it is, how many people you have in it, and your
mining ability.  Also generates revenue in the form of taxes.
\item
Jewel mine (\kbd{j}) - supplies your country with jewels, proportional to
how good a jewel mine it is, how many people you have in it, and your
mining ability.  Also generates revenue in the form of taxes.
\item
City (\kbd{c}) - generate a lot of revenue in the form of taxes.
Cities are also the places in which you can draft armies and construct
ships.  Cities also contain universities and temples, so mages can be
initiated in them.
\item
Capital (\kbd{C}) - are like cities, but the administrative bureaucracy
of your nation is based in your capital, so if your capital is
sacked, many of your nation's riches will be taken.
\item
University (\kbd{u}) - this is a school of higher education.  Your
country's intelligence can be increased if you put a lot of people in
universities.  Universities cost to maintain.  Mages can be initiated
in universities.
\item
Temple (\kbd{+}) - a place of worship.  Mages can be initiated in
temples.  Also, the fraction of your people in temples increases your
magic skill.
\item
Stadium (\kbd{s}) - your country holds sports events here.  (for now,
does nothing)
\item
Trade post (\kbd{T}) - a caravan or ship can give goods and armies to
another nation by dropping them off at a trade post belonging to that
other nation.
\item
Embassy (\kbd{e}) - necessary to establish relations with another
country.  (for now, does nothing)
\item
Fort (\kbd{!}) - forts give bonus to armies camped there (3/turn).
\item
Hospital (\kbd{h}) - hospitals affect birth and death rates in your
nation. (not yet) Hospitals have maintainance costs each turn.
\item
Refinery (\kbd{r}) - refines your metals.  Increases the productivity
of adjacent metal mines.  A metal mine has 12% more production for
each surrounding refinery, if that refinery has at least the minimum
employment.
\end{itemize}

The \emph{Designation table} describes the properties of various
designations: what they cost, how much revenue they produce per
capita, how much they cost to maintain, the minimum employment (not
used in all cases), and how many people can be employed in that
sector.
\begin{same}
\begin{table}[hbpt]
\caption{Designation table}
\begin{tabular}{ || l | l | l | l | l | l | l || }
\hline
Designation &mark&desig& revenue & maint. & min  &  max    \\
            &    &cost & per cap.&per turn&people&employed \\
\hline
None        &  x & 1000 &   30    &    0   &    0 &   20 \\
Farm        &  f & 5000 &  100    &    0   &   10 &  500 \\
Metal mine  &  m &10000 &  100    &    0   &   10 &  800 \\
Jewel mine  &  j &10000 &  100    &    0   &   10 &  800 \\
City        &  c &30000 &  200    &    0   &  300 & 5000 \\
Capital     &  C &50000 &  300    &    0   &  500 & 7000 \\
University  &  u &10000 &   30    & 2000   &  200 & 1000 \\
Temple      &  + & 5000 &    0    &    0   &  200 & 1000 \\
Stadium     &  s & 5000 &  150    &    0   &   10 &  400 \\
Trade post  &  T & 5000 &  150    &    0   &   10 &  300 \\
Embassy     &  e & 5000 &    0    &    0   &   50 &  200 \\
Fort        &  ! &10000 &   50    &    0   &   10 &  200 \\
Hospital    &  h &10000 &  100    & 4000   &   10 &  300 \\
Refinery    &  r & 8000 &  130    &    0   &  100 &  200 \\
\hline
\end{tabular}
\end{table}
\end{same}

\section{Economy and natural resources}
The pillars of your economy are money and natural resources (soil
fertility, metal and jewels).

\subsection{Money}
You get money by levying taxes.  You set the tax rate in the budget
screen.  It is not wise to set a very high tax rate, because that will
kill the spirit of entrepreneurship in your nation, and production of
food, metal and jewels will decrease in proportion.

Money is spent for just about everything: redesignating sectors,
drafting and maintaining armies, research and development, supporting
universities and hospitals, and many other things.

\subsubsection{Debt}

[NOTE: the section on bonds described here is not yet implemented;
for now a debt simply means negative money]

If your nation's money balance goes negative, you will be forced to
issue bonds to your population to finance the debt (this happens
automatically over the update, so there is no way you can plummet into
a negative balance).  You must then pay an interest on these bonds.

At any time you can also negotiate for other nations to purchase your
bonds.  Interest rate on domestic bonds is fixed (15%/thon), but you
can negotiate the price for bonds issued to other countries.

If you want to finance a big war, and get lots of cash fast, the best
way to go is probably to issue a lot of bonds to other countries.

(??) The bonds your nation issues must always be backed by your
reserve of jewels.

\subsection{Soil fertility}
Your nation must produce the food necessary to feed its people.  This is
done by designating certain sectors to be farms.  These farms will be
more productive if they are on sectors with a better soil parameter.

Your farming skill also determines how productive your farms will be.
You can increase your farming skill with research in technology, because
your nation will discover better tools and methods for farming.

If your food production is insufficient, your reserves will be used.
If those are not enough, you had better arrange to purchase some, or
that part of your civilians who did not get enough food will starve.
Once all your civilians have starved, your armies will start to
starve.

\subsection{Metal}
Metal represents all metals and construction materials used for
practical purposes, such as construction and armaments.  It is found
in metal mines.  The production of each mine is greater if the sector
has a higher metal parameter, and is also increased by your nation's
mining skill.  Your mining skill can increase if you invest in
technology, because your nation will discover better tools for mining
and prospecting.

You can spend your metal in several ways, including drafting armies,
technological R\&D, constructing fortification and roads\dots{}

\subsection{Jewels}
In dominion, Jewels represent all kinds of rare resources, such as jewels,
gold, silver, platinum, pearls, and so on.  Jewels are found in jewel
mines.  How many jewels you produce in a mine depends on the jewels
parameter of that sector, and also on your mining ability (see section
on metal).

Jewels are very important, because they are invested in magical
research, and are used to get spells.  In fact, your spell points
depend only on the amount of jewels you invest in magical R\&D.

Jewels are also important in that they constitute your nation's
reserve that backs up its currency and bonds.  When you issue bonds,
these have to be backed up by jewels, so it is a good idea to save up
some jewels, and not spend them all. (note: bonds are not yet
implemented)

(We must find another use for jewels, so nations have more choices to
make in spending jewels.)

\section{Transportation and trade}
You can trade with other nation, or just transport goods/armies/people
for your own benefit, using \strong{ships} and \strong{caravans}.
Caravans travel on land, whereas ships travel in the water.

Ships and caravans are ``drafted'' as if they were armies, and should
appear in your [l]ist of available armies in the [a]rmy menu.  The
construction and maintainance costs of ships and caravans are
tabulated with those of other army types.  Note that some spirits also
behave as ships and caravas, in that they have the [c]argo flag.
Examples of this are the \emph{flying carpet}, the \emph{ghost ship}
and the \emph{living ship}.

A single cargo hold can only transport a certain amount of goods.  The
unit of weight is \emph{the weight of a single person}, and a caravan
can transport 250 person weights.  A bar of metal weighs 0.1, money
weighs 0.01 for a sheckel, food 0.05, a jewel basket 0.01.  If you
load soldiers, their weight is equal to the weight of the number of
people plus the weight of the metal used in drafting an army that
size.  Any caravan can transport only a single army and a single land
title.  The land title does not have significant weight.

To load a caravan or navy, you select it (with [a]rmy commands) and
then use the [t]ransportation command to [l]oad goods, which can be
[s]hekels (money units), [m]etal, [f]ood, [p]eople, [a]army or
[t]itle.  To unload it you move the caravan or navy to its destination
and do the same with [u]nload instead of [l]oad.

You can only load certain goods onto a caravan in certain places.
Anything can be loaded in a city.  Metal can also be loaded in metal
mines, jewels in jewel mines and food in farms.  The title to a sector
\emph{must} be loaded on that sector itself.  People can be loaded
from any of your own populated sectors.  Armies can be loaded anywhere
in your land, but out of your territory they can only be loaded if
they have the front-line flag.

To unload in a foreign land you must be in a trading post, or you can
unload armies with the front-line flag (for example, Sailors, Marines,
Scuba divers).  To trade an army in a foreign trading post, you should
put the army in TRADED status, and then unload it on the trading post.
Alternatively, since armies move on their own, you can put them on
TRADED status and just walk them up to the trading post.  When you
stop that army on the foreign trading post, you will be asked if you
really want to trade it.  You can also change the army to TRADED
status once it is already on the trading post.  In all cases you will
be asked for confirmation of the army trading.

Transporting goods within your country is kind of useless.  In your
land, goods must be unloaded in trading posts or cities.  Transporting
people is an effective way of getting them to the better mines and
farms.  Transporting armies can help mobilize your forces more
quickly, since you can then unload them and they can still march.

Some armies, such as Sailors and Marines, have ``front line'' flag
(see the army types table).  This means that they can be unloaded from
caravans and ships anywhere: in your land, in un-owned land, and in
foreign land.  Thus ships and caravans can be used for transportation
of fighting troops, not just for trade and migration.

\section{Communications}
Nations in dominion communicate through \strong{mail} and \strong{news}.  Mail
allows you to send personal messages to leaders of other nations.
News is for general announcements, and is read by all nations.

There can be several newsgroups.  One is always reserved for messages
from the computer, containing general information on what has happened
over the update.  This newsgroup is usually called ``News''.  Other
newsgroups are set up by the Game Master, and any nation leader can
post to them.  At Stony Brook we usually have a newsgroup called
``public'' which receives many very creative postings from
participants.

You should read your mail and news whenever you play your turn, to be
in touch with your neighbours and the rest of the world.  You also get
mail from the update program after each update, telling you what has
changed in your nation over the update.

\section{Construction}
You can construct on a sector with the [C]onstruct command.  Your
construction will cost money and/or metal, and can make that sector
more valuable.  You can construct:
\begin{itemize}
\item
\emph{[r]oads:} decreases the move cost for you (and anyone else) in this
sector.  The cost for building roads doubles for the next level of
road construction.  For each level of roads construction, the move
cost goes down by 1 (but it never goes below 1).
\item
\emph{[f]ortification:} fortifies the current sector: adds 10 to the
fortification level, which gives \emph{your} armies that much bonus
when defending that sector.
\item
\emph{[b]ubbles:} these are air/water-tight bubbles.  They are necessary
to colonize underwater (if you are an above-water race) and land (if you
are an underwater nation).  Once you have a bubble, you can move
troops and civillians to that sector.
\end{itemize}

\section{World topology}
The world is shaped like a torus (i.e. the surface of a doughnut).
This is the best approximation of a sphere which can be displayed
easily on a flat terminal.

Thus, the world wraps; so, for example, if you are playing
in a 100x100 world, and your nation grows to be 100 sectors 
wide, you can travel around the world.

%\section{Terrain}

\chapter{Diplomacy and war}

Diplomatic relations with your neighbours are extremely important.
Your nation can be destroyed if you did not properly set your
diplomacy: you might make several enemies who could then form a treaty
to fight you.  This happens quite often.

Your nation starts out with 10 armies of 100 Cavemen in your capital;
more can be drafted using the [a]rmy menu.  Soldiers are used to
occupy unowned land, to defend your own territory, and also to conduct
war against enemy nations.

To occupy an unowned sector, you must have an army of at least 100
soldiers there, set on occupy mode.  To occupy a sector owned by
another nation, you have to declare \emph{WAR} or \emph{JIHAD} with
them.  Sectors can also be occupied by spirits with 100 units or more.

\section{Diplomatic status}
In the [r]eports menu, you can access your [d]iplomacy report.  This
report shows you your status toward other countries, and their status
toward you.  You start out \emph{UNMET} with all nations.  Then, as
your armies come close to their sectors, or vice-versa, the two
nations will meet and be put in neutral status.

You can change your status to other countries.  A lot of statuses are
possible, but the most important ones are:
\begin{itemize}
\item[ALLIED]
gives permission to the other nation to pass through your land at a
lower move cost.  Also, you can put your armies in GARRISON in allied
land, and they will get 1/2 of the GARRISON bonus.
\item[TREATY]
goes beyond ALLIED: if your armies or those of the other nation are
involved in a battle, and the other has armies on the same sector, the
two will fight together.  Also, if you put your armies in GARRISON in
treaty land, they will get the full GARRISON bonus.
\item[WAR]
If any army of yours is on the same sector as an enemy army, and one
of the two is on ATTACK or OCCUPY mode, there will be a battle.
\item[JIHAD]
For now, the same as WAR.  In the future, JIHAD should involve some
expense, and give a better fighting bonus due to fanatism in combat.
\end{itemize}

You can change your diplomatic status towards any nation you have met.
They will see the change immediately.  You can only change it by two
degrees, so that you cannot be ALLIED, march into someone's land, and
then declare WAR and occupy all their sectors.

\section{Armies}
The \emph{army types} table lists the various types of armies, their
costs, bonuses and move rates.

Each army or spirit has a set of \strong{army flags} that affect its
behaviour.  Some are innate abilities of that army, others can be set
with magic spells.  The table lists each army's innate flags.  The
flag abbreviations are:
\begin{itemize}
\item
F: Flying.  Any army with this flag can fly.  Move costs of sectors
don't depend on altitude, and thus are much lower.  These armies also
ignore PATROLs and INTERCEPTs unless the patrolling army has missiles.
\item
H:  Hidden.  Army with this flag is magically cloaked.
\item
V:  Vampire.  Army with this flag will possess some of the dead
on the battlefield who will join ranks with this army.
\item
T:  Transport.  This army is being transported as cargo.
\item
\^:  Missiles.  This army has weapons that are thrown, such as arrows.
Archers are a type of army with mssles.  These units, when in PATROL
or INTERCEPT status, will slow down flying units too (normally flying
units are not slowed down by PATROLs or INTERCEPTs, though they
\emph{do} get intercepted once they land).
\item
W:  Water.  An army with this flag must stay in water, or in the land
sectors along a coast; even if it belongs to a land nation.  Compare
with the L flag below.
\item
f: Front-line army.  Can be loaded/unloaded on/from a ship or caravan on
land that does not belong to you.
\item
k:  Kamikaze.  Armies with this flag will fight with very high bonus,
but will all die in any battle.
\item
m:  Machine.  These armies are really machines, such as siege engines,
war carts and catapults.  They help ordinary armies in combat by
destroying fortifications and adding bonus to the armies.
\item
d: Disguised.  You can disguise this army so that it will appear to be
of another type to other players.  For example, to disguise a
poltergeist as an areal serpent you would change the armies
\emph{name} so that it ends with ``/areal_serpent''.
\item
w:  Wizard.  This flag means that the army can summon spirits
and cast spells, like a mage.
\item
s:  Sorcerer.  This flag means that the army can excercise powers
of sorcery (not implemented yet).
\item
c:  Cargo.  This army can transport cargo, like ships and caravans.
\item
U:  Underground.  This army can burrow underground, and thus is not
slowed down by patrols and intercepts.
\item
L:  Land.  This is the exact inverse of the W flag.  An army with
this flag must stay on land, or in water sectors along the coast; even
if it belongs to a water nation.  Notice that if you have both the
L and W flags, you can move that army onto \emph{any} sector on the
map.
\item
I:  Inverse altitude.  This flag says that the army will have either
the L or the W flag, depending on the race of the army's owner.  If
the army belongs to a \emph{land} race, then it will have a W flag,
and vice-versa.  This flag is used mostly for things like ships,
which travel in the ``inverse'' altitude medium.
\item
*: Race specific.  These armies are available only to certain
races.
\end{itemize}

\section{Army types}

The \emph{Army types} table below describes in detail all army types
available for drafting.

\begin{same}
\begin{table}[hbpt]
\caption{Army types table}
\begin{tabular}{ || l | l | r | r | r | r | r | r | r | r | l || }
\hline
Type      &char&move&bonus&draft&draft&draft&maint&maint&maint&flags \\
          &    &rate&     &money&metal&jewel&money&metal&jewel&      \\
\hline
Cavemen      &*& 0.5& -60 &  60 &   0 &   0 &  10 &   0 &   0 &\\
Spearmen     &/& 0.8& -10 &  80 &  20 &   0 &  10 &   0 &   0 &\\
Infantry     &i& 1.0&   0 & 100 &  30 &   0 &  10 &   0 &   0 &\\
Caravan      &C& 1.0&   0 &1000 &1000 &   0 & 200 &  40 &   0 &c\\
Archers      &)& 1.0&   5 & 100 &  40 &   0 &  10 &   0 &   0 &^\\
Swimmers     &o& 1.0& -60 & 150 &  50 &   0 &  15 &   5 &   0 &Wf\\
Walkers      &#& 1.0& -60 & 150 &  50 &   0 &  15 &   5 &   0 &L\\
Phalanx      &p& 1.0&  20 & 150 &  50 &   0 &  12 &   0 &   0 &\\
Sailors      &~& 0.0& -10 & 150 &  50 &   0 &  15 &   0 &   0 &f\\
Chariots     &0& 1.5&  10 & 180 &  60 &   0 &  14 &   5 &   0 &\\
Canoes       &u& 2.0& -80 &1000 &2000 &   0 & 200 & 100 &   0 &cI\\
Legion       &l& 1.0&  40 & 200 &  70 &   0 &  16 &   0 &   0 &\\
Cavalry      &c& 2.0&  20 & 200 &  80 &   0 &  20 &   0 &   0 &\\
Elite        &e& 1.3&  10 &   0 & 100 &   0 &   0 &  25 &   0 &\\
Sailboats    &\}& 2.5& -50 &1500 &3000 &   0 & 300 & 150 &   0 &cI\\
Marines      &m& 1.0&  60 & 300 & 200 &   0 &  25 &   0 &   0 &f\\
War_carts    &w& 1.0&  10 & 500 & 200 &   0 & 200 &  20 &   0 &m\\
Galleys      &g& 3.5& -20 &2000 &4000 &   0 & 400 & 200 &   0 &cI\\
Berzerkers   &b& 1.0&  50 &  30 &   0 &   0 &   5 &   0 &   0 &k\\
Merc         &M& 1.2&  60 & 200 &   0 &   0 & 200 &   0 &   0 &\\
Catapults    &@& 0.5&  20 & 800 & 400 &   0 & 300 &  30 &   0 &m\\
Quadriremes  &q& 4.5&   0 &3000 &5000 &   0 & 500 & 250 &   0 &cI\\
Scuba_divers &S& 1.2& -30 & 300 & 200 &   0 &  25 &  25 &   0 &fI\\
Kamikaze     &k& 1.5& 200 &  50 &  25 &   0 &  50 &   0 &   0 &k\\
Wagons       &W& 1.5&   0 &1500 &1500 &   0 & 300 &  60 &   0 &c\\
Crossbowmen  &]& 1.0&   5 &  50 &  60 &   0 &   5 &   5 &   0 &^\\
Ninja        &N& 3.0&  75 &2000 &   0 &2000 & 500 &   0 & 500 &H\\
Mage         &!& 2.0&   0 &   0 &   0 &5000 &   0 &   0 &1000 &w\\
Hunters      &h& 1.5& -10 &  50 &   0 &   0 &  10 &   0 &   0 &\\
Orcs         &O& 1.0&   0 &  60 &  30 &   0 &  10 &   0 &   0 &\\
Harpies      &y& 1.0&  10 & 100 &  30 &   0 &  10 &   0 &   0 &F\\
Hobbits      &H& 1.0&   0 & 100 &  30 &   0 &  10 &   0 &   0 &H\\
Ogres        &G& 1.0&  50 & 100 &  30 &   0 &  10 &   0 &   0 &\\
\hline
\end{tabular}
\end{table}
\end{same}

\section{Spirit types}
The \emph{Spirit types} tables below list the spirit types available in
dominion.  Summoning costs (in spell points) are not listed, because
they can be different if a spirit is available to more than one magic
order.  The flags for each spirit are described in the section on
\emph{Armies} above.

% Made with Titus Brown's mktables

\begin{same}
\begin{table}[hbpt]
\caption{Spirit types table (Diana)}
\begin{center}
\begin{tabular}{ || l | l | l | l | l | l || }
\hline
Name            & Size & Move & Bonus &Spell&Flags\\
                &      & rate &       &Pts  &\\
\hline
wolf            &   30 & 1.5 &   0 &  1 &\\
swarm           &   50 & 1.3 & 300 &  3 &Fk\\
mole            &   70 & 1.0 &   0 &  3 &UL\\
snake           &  100 & 0.8 &   0 &  3 &\\
shark           &  100 & 1.5 &  10 &  3 &W\\
hawk            &  200 & 2.0 &   0 &  5 &F\\
bear            &  300 & 1.4 &   0 &  5 &\\
lion            &  500 & 1.5 &   0 &  6 &L\\
terrasque       & 2000 & 0.5 &   0 & 13 &L\\
\hline
\end{tabular}
\end{center}
\end{table}
\end{same}
\begin{same}
\begin{table}[hbpt]
\caption{Spirit types table (Necromancy)}
\begin{center}
\begin{tabular}{ || l | l | l | l | l | l || }
\hline
Name            & Size & Move & Bonus &Spell&Flags\\
                &      & rate &       &Pts  &\\
\hline
ghost_ship      &    1 & 2.5 &   0 &  1 &HcI\\
skeleton        &   40 & 0.8 &   0 &  1 &\\
wraith          &   30 & 1.5 &   0 &  2 &H\\
zombie          &   70 & 0.8 &   0 &  3 &V\\
poltergeist     &  150 & 0.1 &   0 &  3 &d\\
mummy           &   80 & 0.8 &  30 &  3 &U\\
ghost_ship      &    1 & 2.5 &   0 &  5 &HcI\\
lacedon         &  500 & 1.0 &   0 &  7 &W\\
lich            & 2000 & 2.0 & -20 & 14 &w\\
vampire         & 1500 & 0.8 &  30 & 15 &V\\
\hline
\end{tabular}
\end{center}
\end{table}
\end{same}
\begin{same}
\begin{table}[hbpt]
\caption{Spirit types table (Yavanna)}
\begin{center}
\begin{tabular}{ || l | l | l | l | l | l || }
\hline
Name            & Size & Move & Bonus &Spell&Flags\\
                &      & rate &       &Pts  &\\
\hline
dryad           &   30 & 1.0 &   0 &  1 &\\
tree_spirit     &   50 & 1.5 &   0 &  2 &F\\
wood_beast      &   70 & 1.0 &   0 &  2 &\\
magic_trees     &  150 & 1.0 &   0 &  3 &\\
yellow_musk     &  200 & 0.7 &   0 &  4 &V\\
ent             &  250 & 0.8 &   0 &  4 &\\
swamp_beast     &  500 & 1.0 &  10 &  6 &W\\
shambling_mound & 1200 & 0.8 &  10 & 10 &\\
forest          & 2000 & 0.7 &   0 & 14 &\\
\hline
\end{tabular}
\end{center}
\end{table}
\end{same}
\begin{same}
\begin{table}[hbpt]
\caption{Spirit types table (Demonology)}
\begin{center}
\begin{tabular}{ || l | l | l | l | l | l || }
\hline
Name            & Size & Move & Bonus &Spell&Flags\\
                &      & rate &       &Pts  &\\
\hline
imp             &   30 & 2.0 &   0 &  2 &\\
lesser_demon    &   70 & 1.7 &   0 &  3 &\\
hellhound       &  150 & 1.7 &   0 &  4 &\\
tormented_soul  &  300 & 1.2 & 100 &  5 &k\\
devil           &  250 & 1.0 &   0 &  4 &\\
major_demon     &  500 & 2.0 &   0 &  8 &F\\
balrog          & 1500 & 1.5 &  50 & 15 &\\
demon_lord      & 2500 & 2.0 &   0 & 19 &w\\
\hline
\end{tabular}
\end{center}
\end{table}
\end{same}
\begin{same}
\begin{table}[hbpt]
\caption{Spirit types table (Monsters)}
\begin{center}
\begin{tabular}{ || l | l | l | l | l | l || }
\hline
Name            & Size & Move & Bonus &Spell&Flags\\
                &      & rate &       &Pts  &\\
\hline
spider          &   30 & 0.8 &   0 &  1 &\\
yeti            &   70 & 1.0 &   0 &  2 &L\\
ettin           &  120 & 1.5 &   0 &  3 &\\
cyclops         &  150 & 0.8 &   0 &  3 &\\
hydra           &  250 & 1.2 &   0 &  4 &\\
crimson_death   &  500 & 2.0 &   0 &  8 &F\\
sea_dragon      & 1000 & 1.2 &   0 &  9 &W\\
green_dragon    & 1500 & 2.0 &   0 & 11 &L\\
red_dragon      & 2500 & 1.5 &   0 & 17 &F\\
gold_dragon     & 3500 & 2.0 &   0 & 24 &Fw\\
\hline
\end{tabular}
\end{center}
\end{table}
\end{same}
\begin{same}
\begin{table}[hbpt]
\caption{Spirit types table (Neptune)}
\begin{center}
\begin{tabular}{ || l | l | l | l | l | l || }
\hline
Name            & Size & Move & Bonus &Spell&Flags\\
                &      & rate &       &Pts  &\\
\hline
living_ship     &    4 & 2.5 &   0 &  1 &cI\\
pirana          &   30 & 1.2 &   0 &  1 &W\\
water_nymph     &   50 & 1.2 &   0 &  2 &W\\
whale           &   80 & 1.0 &   0 &  2 &W\\
shark           &  100 & 1.5 &  10 &  3 &W\\
sea_serpent     &  150 & 1.2 &   0 &  4 &WL\\
craken          &  250 & 1.0 &   0 &  4 &W\\
sea_giant       &  500 & 0.8 &   0 &  5 &W\\
water_elemental & 2000 & 2.0 &   0 & 15 &W\\
leviathan       & 3000 & 0.9 &  10 & 18 &W\\
\hline
\end{tabular}
\end{center}
\end{table}
\end{same}
\begin{same}
\begin{table}[hbpt]
\caption{Spirit types table (Aule)}
\begin{center}
\begin{tabular}{ || l | l | l | l | l | l || }
\hline
Name            & Size & Move & Bonus &Spell&Flags\\
                &      & rate &       &Pts  &\\
\hline
gargoyle        &   30 & 1.0 &   0 &  2 &F\\
mole            &   70 & 1.0 &   0 &  3 &UL\\
dust_devil      &  150 & 1.0 &   0 &  4 &FL\\
umber_hulk      &  250 & 1.7 &   0 &  5 &UL\\
stone_giant     &  500 & 0.7 &   0 &  6 &\\
earth_elemental & 2000 & 2.0 &   0 & 16 &U\\
mountain        & 3000 & 0.3 &  60 & 23 &\\
\hline
\end{tabular}
\end{center}
\end{table}
\end{same}
\begin{same}
\begin{table}[hbpt]
\caption{Spirit types table (Chess)}
\begin{center}
\begin{tabular}{ || l | l | l | l | l | l || }
\hline
Name            & Size & Move & Bonus &Spell&Flags\\
                &      & rate &       &Pts  &\\
\hline
pawn            &   30 & 0.5 &   0 &  1 &\\
knight          &  120 & 1.0 &  10 &  4 &H\\
bishop          &  150 & 2.0 &   0 &  5 &F\\
rook            &  250 & 3.0 &  30 &  6 &\\
queen           &  500 & 4.0 &  50 &  9 &\\
king            & 1000 & 0.4 &   0 &  8 &\\
master          & 2000 & 1.0 &  10 & 11 &\\
grandmaster     & 3000 & 1.3 &  20 & 14 &w\\
\hline
\end{tabular}
\end{center}
\end{table}
\end{same}
\begin{same}
\begin{table}[hbpt]
\caption{Spirit types table (Inferno)}
\begin{center}
\begin{tabular}{ || l | l | l | l | l | l || }
\hline
Name            & Size & Move & Bonus &Spell&Flags\\
                &      & rate &       &Pts  &\\
\hline
efreet          &   70 & 1.5 &   0 &  2 &L\\
phoenix         &   30 & 2.0 &   0 &  3 &FL\\
fire_giant      &  250 & 1.0 &   0 &  4 &L\\
fire_drake      &  500 & 1.5 &   0 &  6 &\\
lava_beast      & 1000 & 1.0 &   0 & 10 &U\\
fire_elemental  & 2000 & 2.0 &   0 & 15 &L\\
conflagration   & 3000 & 1.5 &   0 & 18 &L\\
\hline
\end{tabular}
\end{center}
\end{table}
\end{same}
\begin{same}
\begin{table}[hbpt]
\caption{Spirit types table (Avian)}
\begin{center}
\begin{tabular}{ || l | l | l | l | l | l || }
\hline
Name            & Size & Move & Bonus &Spell&Flags\\
                &      & rate &       &Pts  &\\
\hline
flying_carpet   &    1 & 1.5 &   0 &  1 &Fc\\
roc             &   30 & 1.5 &   0 &  1 &F\\
eagle           &   70 & 1.5 &   0 &  2 &F\\
cloud_giant     &  150 & 1.0 &   0 &  3 &F\\
wyvern          &  250 & 1.0 &   0 &  4 &F\\
areal_serpent   &  500 & 1.8 &   0 &  7 &F\\
air_elemental   & 2000 & 2.0 &   0 & 15 &F\\
tempest         & 3000 & 2.0 &   0 & 18 &F\\
\hline
\end{tabular}
\end{center}
\end{table}
\end{same}
\begin{same}
\begin{table}[hbpt]
\caption{Spirit types table (Unity)}
\begin{center}
\begin{tabular}{ || l | l | l | l | l | l || }
\hline
Name            & Size & Move & Bonus &Spell&Flags\\
                &      & rate &       &Pts  &\\
\hline
naga            &   30 & 1.0 &   0 &  1 &\\
centaur         &   70 & 2.0 &   0 &  3 &\\
werewolf        &  100 & 1.5 &   0 &  4 &Vd\\
minotaur        &  150 & 1.0 &   0 &  3 &L\\
owl_bear        &  250 & 1.0 &   0 &  4 &\\
gryphon         &  350 & 2.0 &   0 &  7 &F\\
sea_lion        &  600 & 1.5 &   0 &  7 &W\\
chimera         & 1000 & 1.5 &   0 & 10 &F\\
sphynx          & 2000 & 0.8 &   0 & 17 &Fw\\
\hline
\end{tabular}
\end{center}
\end{table}
\end{same}
\begin{same}
\begin{table}[hbpt]
\caption{Spirit types table (Time)}
\begin{center}
\begin{tabular}{ || l | l | l | l | l | l || }
\hline
Name            & Size & Move & Bonus &Spell&Flags\\
                &      & rate &       &Pts  &\\
\hline
diplodocus      &   30 & 1.3 &   0 &  1 &\\
brontosaurus    &   70 & 0.7 &   0 &  2 &\\
pleisiosaurus   &   50 & 1.5 &   0 &  2 &W\\
pterodactyl     &  150 & 1.5 &   0 &  4 &F\\
stegosaurus     &  250 & 1.0 &   0 &  4 &\\
triceratops     &  500 & 0.8 &   0 &  6 &L\\
tyrannosaurus   & 1500 & 1.5 &   0 & 12 &\\
brachiosaurus   & 3000 & 0.5 &   0 & 18 &WL\\
\hline
\end{tabular}
\end{center}
\end{table}
\end{same}
\begin{same}
\begin{table}[hbpt]
\caption{Spirit types table (Insects)}
\begin{center}
\begin{tabular}{ || l | l | l | l | l | l || }
\hline
Name            & Size & Move & Bonus &Spell&Flags\\
                &      & rate &       &Pts  &\\
\hline
ant             &   30 & 1.0 &   0 &  2 &UL\\
swarm           &   50 & 1.3 & 300 &  3 &Fk\\
flying_carpet   &    1 & 1.5 &   0 &  3 &Fc\\
moth            &  150 & 1.5 &   0 &  4 &F\\
bee             &  250 & 1.2 &  30 &  6 &F\\
grasshopper     &  350 & 1.0 &   0 &  6 &HL\\
mosquito        &  500 & 0.5 &   0 &  6 &FV\\
infestation     &  600 & 0.0 &  50 &  6 &\\
roach           & 1000 & 0.8 &  20 & 10 &\\
lobster         & 1500 & 1.0 &   0 & 12 &W\\
creeping_doom   & 2500 & 0.8 &  20 & 18 &\\
\hline
\end{tabular}
\end{center}
\end{table}
\end{same}



\section{Army maintainance costs}
Each army has a maintainance cost which is given by the salaries of
the soldiers plus a fixed per-army overhead of 2000 sk.

\section{Movement points}
Your armies can move only so much before they must stop and rest.  How
much they can move is expressed in the army's \strong{move points}, and
the \strong{move cost} of the sectors they must cross.

For example, if an army has 13 move points, and it goes through
sectors with move costs of 4, 3, 2, 2 and 3, it will not have enough
move points left for the last sector (it will have 2 move points left,
and that last sector has a move cost of 3).

\section{Army statuses}
\begin{itemize}
\item
\emph{[i]ntercept} This army will intercept a nearby army.  Intercept
raises the move cost in the area, and then (over the update) the army
will move to any adjacent square where an enemy army might be located.
This is quite powerful, since you don't have to defend each square;
but you should be careful because intercepts can be decoyed.

Intercept will also move the army to intercept a flying enemy army,
once it has landed.  But the increased move cost will not apply to flying
enemy armies \emph{unless} the intercepting army wields missiles (for
example, archers).  See the description of army flags.

\item
\emph{[p]atrol} This army will patrol the area, raising the move cost
for enemy armies.  If the enemy army is flying, the move cost will
only be raised if the patrolling army has missiles (for example, archers).

\item
\emph{[g]garrison} This army garrisons the sector, getting an
extra 10 bonus on your own land.  If the sector belongs to a nation at
TREATY with you, then you also get bonus 10.  If it is ALLIED land,
you get a bonus of 5.
\end{itemize}

\comment \section{Tactics}

\chapter{Detailed description of commands}

\section{Options menu}
You can set some options that affect the way the game appears to you.
For now the only ones implemented are:
\begin{itemize}
\item
\emph{e[x]pert} mode.  This toggles expert mode on and off.  Expert mode
allows an experience player to do things much more quickly by showing
most menus on the status line, instead of drawing big windows.
\item
\emph{[f]orwarding mail}.  This allows a you to get mail forwarded
to your account instead of reading mail inside dominion.
\item
\emph{[c]ivilian movement}.  This toggles between the three available
types of migration that your government allows: \strong{Free},
\strong{Restricted} and \strong{None}.
\item
\emph{[m]ail program}.  This allows you to choose a mail program with
which to read your dominion mail.  Examples are ``elm'', ``mush'', ``Mail'',
``mailx''.  If you type nothing, you will get the builtin bare-bones mail
program.
\end{itemize}

\section{Display options}
The default way of viewing the screen shows the designation of your
own sectors, altitude markings for unowned sectors, and nation marks
for other nations' sectors.  But there are a lot of different ways of
looking at the map, and of highlighting sectors on it.  The [d]isplay
command will list all these for you.

If you are a land race, water sectors are usually marked as a '~', but
this can be toggled with the \emph{[W]ater toggle} option: if you
choose this mode, then land sectors will appear as a '.', and water
sectors will be shown in detail.

\section{Movement}
The movement commands are shown in the diagram at the beginning of this
manual.  They are quite straightforward, and only behave differently
when you reach the edge of the screen.  In this case, the screen is
shifted over, so your cursor does not go off.  The screen can be
forcibly centered around the cursor using the [d]isplay command [C].

You can also jump directly to a sector with the [p] command.  You will
be prompted for the coordinates.  You can jump back to your capital
with the [P] command.

\section{Reports}
Here is a detailed description of each report in dominion. To acess
them type [r] and then the letter of the report you want.  You can
switch from one report to another by hitting the key that corresponds
to the report you want.  The reports you can switch to ae listed at
the bottom of the report you are looking at.

\subsection{Information report}
To access this report you type [r] followed by [i].

This report gives complete information about your nation.  This is a
description of each parameter printed.

\begin{itemize}
\item
Nation name: the name of your nation (wow!!).
\item
Nation id: your nation id, used mostly internally.
\item
Leader: the name of your nation's leader.  
\item
Capital: the location of your capital.  This is always (0,0), since
coordinates are given relative to your capital, unless you are the game
master.
\item
Race: the race you belong to.
\item
Mark: your nation mark.  Other nations will see this mark on their screen,
and you will too if you display by nation mark.
\item
Secotors: The number of sectors owned by your nation.
\item
Treasury, Jewels, Metal, Food: Your current wealth.
\item
Birthrate: The percentage by which your population increases every thon.
\item
Mortality: The percentage of people who die in your nation every thon.
\item
Inteligence:
\item
Magic Apt:
\item
Speed: Your move rate is proportional to this.
\item
Initiated to the magical order:  This says what magic order you belong to.
\item 
Magic Skill: How proficient you are in your use of magic.
\item
Spell points: Tells you how many you have and how many are spent for 
maintenance of spirits.
\item
Technology skill: How technologically advanced your country is.
\item
Farming skill:  The higher this is the more food you produce per farmer.
\item
Mining skill: This is your ability to 'harvest' jewels and metal.
\item
Spy skill: This expresses how good your intelligence service is.
\item
Secrecy: This makes you immune to espionage by other nations.
\item
Civilians: This is the number of civilans in your country.
\item
Soldiers: This is the number of soldiers in your country.  This
includes spirit forces, mages, caravans, ships...
\item
Armies: This is the number of armies you have including spirit units
and mages.
\item
Attack bonus, Defense bonus: If you add this percentage to the number
of soldiers in the army, you get the effective force your army fights
with.
\item
Move points:  This is how far your army can move each turn.  
\end{itemize}

From here you can change your password [p] or change your leader name
[l].  The key letter [t] allows you to change your nation to an NPC,
and allows you to choose if your nation should receive update mail
while it is being run by the computer.  Also listed are the key
letters for other reports you can view.

Changing your nation to an NPC is a risky move: the computer will play
a good game, but it will not honor your long-term plans.

\subsection{Budget report}
To access this report you type [r] followed by [b].  

The budget report gives you detailed breakdown of how you are spending your 
money and your natural resources.  

Within the budget report you can adjust what percent of your money
and/or natural resources you are spending on technology, the study of
magic and reconnaisance. You also can adjust your tax rate from this
menu.

This screen also shows how much money you are spending on military
maintenance and other costs inccurred. The only way military
maintenance can be lowered is to disband armies. Other costs include
but are not limited to the cost of drafting an army and the cost of
redesignating sectors.

The Metal and Jewels Breakdown, aside from listing the amounts spent
on research and development, show how much of each natural resource is
spent.  Other metal expenses consist of but are not limited to the
construction of citites and the drafting of armies. Other Jewels are
used as a maintenance fee for mages.

The commands within this screen are:

\begin{itemize}
\item
{[t]} which adjusts your tax rate
\item
{[c]} which adjusts your charity rate
\item
{[T]} which is technology resarch and development, has a
submenu that consists of the choices metal and money.
\begin{itemize}
\item
   {[m]} sets the amount of metal you wish to use in R\&D 
\item
   {[M]} sets the amount of money
\end{itemize}
\item
{[M]} puts you into the sub-section for Magic R\&D where
\begin{itemize}
\item
   {[j]} changes the amount of jewels you use for Magic R\&D
\item
   {[M]} adjusts the amount of money you spend for Magic R\&D
\end{itemize}
\item
{[S]} this invests money in your \emph{spy department}
\item
{[s]} this allows you to spend money/metal/jewels from your storage.
The way you can spend it is similar to how you spend your revenue, but
it is a once-only expense, and is cleared after each update.
\end{itemize}

\emph{A NOTE OF CAUTION:}

Watch how much money you are spending carefully, You might be
bankrupting your nation without realizing it!  You to can run your
nation at a deficit\dots{} but be warned:  if you have no money, you
canot draft, construct or redesignate.  If someone attacks you and
you need armies quickly, you will be in trouble.

You might also want to keep you taxes on the low side\dots{} the
higher the taxes, the less the people produce.  In fact, beyond a
certain tax rate, you will not even get much tax revenue out of your
people because they will produce so little, and they will want to
cheat on taxes.

\subsection{Production report}
The production report tells you how many people are employed and
unemployed in each area of your economy.  It also shows you the
average productivity of each employee.  The ``service sector''
means all people who are not employed in productive endevours (mining
and farming).

\subsection{Nations report}
The nations report, which can be seen by typing [r] followed by [n],
gives you a report on all the nations.

You are told the size of the world and how many nations are in it. It also 
gives a detail of each nation that lists the nation id, nation name, nation 
mark, leader name, and race.

Within this report is also the spy option, [s].  You will then be
prompted for a nation id, and then you will be presented with a screen
that allows you to bribe officials in that nation for information.
Here is how it works:

You pay a certain amount of jewels in bribes to another nation.  You
can get information about their [p]opulation, [e]conomy, [m]ilitary,
ma[g]ic, [t]echnology and [C]apital location.  You can also steal
[T]echnology (not yet implemented).

The success of your spying will depend on your spy level, your
opponent's secrecy, your and your opponent's stealth, and the amount
of jewels you spend in bribes.  You will receive an answer which is
not exactly accurate, but gets better if you spend more, or have a
better spy value, and so on.

\subsection{Diplomacy report}
Diplomacy with other nations is extremely important.  You should set a
status with each nation you have met.  This is done in the
\strong{diplomacy} report.

You cannot make a great change in diplomacy status in one turn: you
can at most jump two levels of diplomacy in one turn, like going from
ALLIED to RECOGNIZED.  The next turn you can go from RECOGNIZED to
UNRECOGNIZED, and the turn after that you can go do WAR.  This ensures
that any nation will see the diplomatic situation deteriorate
progressively before war is declared on them.

\section{Wizardry}
Type [W] to enter the wizardry menu.

The wizardry command allows you to do \strong{initiate a mage},
\strong{cast a spell} and \strong{summon a spirit}.  Spells are cast and
spirits are summoned in a given sector, and you must have a mage in
that sector to do so.

Hit the [i] command to initiate a mage.  Mages move twice as fast as
normal armies, so they can reach a battlefield quickly to do their
work.

If you have a mage already selected (see the [a]rmy menu), then you
can use use [c] to [c]ast a spell or [s] to [s]ummon a spirit.  The
spell is cast on an object (army, or sector, or whatever, according to
which spell it is) in the same sector as the mage, and the spirit is
created in the same sector as the mage.  Spirits must be summoned in
your own land.

Most spells will hang around until their duration is over, or until
you delete them, or until the army to which they apply does not exist
any more.  You can see which spells you have hanging with the [h]
command, and that gives you the option of deleting hanging spells too.
Spirits last until they are killed, and behave just like armies.

\subsection{Mages}
Mages are necessary for casting spells and summoning spirits.  They
must be initiated inside a capital, city, temple or university.  They
cost 5000 jewels to initiate, and 1000 jewels in maintainance each
thon.

Mages are moved around as if they were an army, and have twice the
nation's basic move rate.

Some spirits have the [w]izardry flag, and they behave like mages.

\subsection{Spells}
Here is a description of several spells available in dominion.  Keep
in mind that some spells may be available to several orders, and their
cost and duration will be different for the different orders.  For
example, both Neptune and Time have the water_walk spell, but for
neptune it costs only 1 spell point, whereas for Time it costs 2.

Both the cost and the duration of the spell are indicated when you
list your available spells.  The cost is in spell points.

If the spell is applied to a sector, then that is all you spend.

A spell applied to an army will usually set a special flag for that
army, and will cost a given amount per each 100 men.  Army flags are
described in the section on armies.

Notice that some armies and spirits come into the world with some of
these magical properties already set, so you do not need to set them
with a spell.

\begin{itemize}
\item \strong{caltitude}
This spell allows you to raise or lower the altitude of a sector.
This means that you could plunge it into the sea, or you could place
it on a mountain peak.
\item \strong{fertility}
This spell allows you to raise or lower the soil fertility of a
sector.
\item \strong{cmetal}
This spell allows you to raise or lower the metal productivity of a
sector.
\item \strong{cjewels}
This spell allows you to raise or lower the jewel productivity of a
sector.
\item \strong{fireburst}
This spell devastates the chosen sector, redesignates it to
\emph{none}, sets the soil productivity to zero, and in the future
should do other nasty stuff.
\item \strong{inferno}
This spell will kill all population in the sector, and make it
completely impenetrable to any army for its duration.  You can only
cast \emph{inferno} on your own sectors.  It is a very effective
way to block enemy armies.
\item \strong{hide_sector}
This spell completely hides the current sector from any other nation.
The sector will just appear as a blank spot on the map.
\item \strong{hide_army}
This spell sets the HIDDEN flag on an army, so that others cannot see
this army.
\comment  All they will see is a highlighted spot on their display
\comment if they highlight by armies, but when they try to examine the sector,
\comment they will just see a mist and have no idea whose armies they are, nor
\comment how many men are in it.
\item \strong{fly_army}
This sets the FLYING flag on an army.  The army will be able to fly.
\item \strong{vampire_army}
This sets the VAMPIRE flag on an army.  This means that in battle, 1/4
of the troops that die will join the ranks of this army.  The army
will not be allowed to grow to more than a certain multiple of its
original size.
\item \strong{burrow_army}
This sets the UNDERGROUND flag on an army.  This means that the army is 
traveling underground and is thus immune to intercepts.
\item \strong{water_walk}
This sets the WATER flag on an army.  The army will be able to cross
water sectors.
\item \strong{mag_bonus}
This gives magical enhancement to an army.  The army will fight with an
extra 30% bonus.
\item \strong{merge}
This spell allows you to merge civilians into spirits (up to twice the
spirits' basic strength), or to take spirits and merge them into the
civilian population of a sector.
\item \strong{haste_army}
This spell doubles the army's current move rate.
\end{itemize}

\subsection{Spirits}
Spirits are like armies, in that they can fight, and they can occupy
sectors (if they are big enough), and their status and movement is
manipulated with the [a]rmy command.

They are also different in many ways.  To obtain them, you don't draft
them, but your mage summons them with spell points.  They are not
maintained by money, but by more spell points (typically 1/3 of the
spell points that were needed to summon them in the first place).

Spirit types are described above together with army types.

\chapter{The update}

It can be useful to understand exactly what happens during the update.
Here are the steps in the order in which the computer performs them.
This list might not be complete, but it should give you an idea of how
the update works.

\begin{enumerate}

\item NPC diplomacy is updated
\item Hanging spells are loaded
\item All moves by all nations are incorporated, and NPC moves are made
\item Technology is updated
\item Spy is updated
\item Magic is updated
\item Revenue is calculated of money, metal, jewels and food
\item Civilian migration occurs
\item Battles are resolved
\item Sector capture is handled
\item Diplomacy is updated (nations become ``met'')
\item Armies are reset.  Mages are disbanded if maintaince jewels are missing.

\end{enumerate}

\section{Migration}

Civilian migration happens automatically.  The people move according to
the laws of the country.  As a leader you can set those laws with the
[o]ptions screen:  you can set migration to be \emph{Free} (the default),
\emph{Restricted} or \emph{None}.

\begin{itemize}
\item Free
movement.  People look for the sectors that have the most
available jobs, because that's where they will get the best pay/best
job.  Some attraction is also excersized by how pleasant a sector's
living conditions are, but in general a constant ratio of employment
is preserved locally.
\item Restricted
movement.  The govevernment (you) has gotten sick of all theese
civilians wandering around so much.  So civilians will not be allowed
to leave their present place of residence unless they can prove they
are unemployed.  If so then they will move to the sectors surrounding
them, partially by number of jobs available, partially by the
desirability of the sectors available.  However all sectors in range
will be put to full employement if possible.  If there is nowhere to
go they will wander aimlessly in search of a place to find a job,
jumping with the latest rumor of employment.  However even in the best
circumstance, due to a slow bureocracy there will always be a few
extra people left in the sector than can be employed.
\item None.
No movement.  The government (you again) has decreed that there will
be no civillian movement.  Where they are is where you stay, and where
where they are born is where they will die.  Unless the army comes to
get them in caravans/ships, they will stay put.
\end{itemize}

\chapter{Authors}
Version 1.02 of Dominion is the first with this name.  It used to be
called \emph{Stony Brook World} (SBW), until too many people suggested
a catchier name.

Here is a list of the people who actually wrote code for sbw/dominion
that is in the current release.  The order is that in which they wrote
their first piece of code.

Mark Galassi (rosalia@dirac.physics.sunysb.edu) User interface (in
curses), basic data structures, world generator, economy, technology,
magic, basic army work, manual and formatting of manual with
LaTeXinfo, miscellaneous.  Currently maintains dominion.

Michael Fischer (greendog@max.physics.sunysb.edu) Update program, trade
board, developed exec file format, miscellaneous.

Doug Novellano (doug@max.physics.sunysb.edu) Mail and News systems.

Keith Messing (keith@max.physics.sunysb.edu) Diplomacy system.

Alan Saporta (gandalf@max.physics.sunysb.edu) Work on some exec routines,
many suggestions of directions for the game.

Joanne Rosenshein (raven@max.physics.sunysb.edu) Initial draft of the
manual, many suggestions of directions for the game.

Stephen Bae (sbae@max.physics.sunysb.edu) Basic world memory allocation.

Chris Coligado (noel@max.physics.sunysb.edu) Army and battle code.

C. Titus Brown (brown@dirac.physics.sunysb.edu) Adding nations and
improvements on the reports; revised army menu and transportation
menu.  Lots of miscellaneous stuff.

Charles Ofria (charles@max.physics.sunysb.edu) Designed the ``npcs''
file, most magic orders and many races and techno powers; coded some
spells.

There are also some contributions from people not in Stony Brook:

Stephen Underwood (su11+@andrew.cmu.edu) Fractal terrain generator and
contributions in very many areas, including the standalone mail
reader.

Paolo Montrasio (montra@ghost.unimi.it) .dominionrc parser (for
a later release), and working on design for distributed game.

Kevin Hart (hart@susan.cs.andrews.edu) NPC system.

Many others have made very important creative suggestions to the game,
though they were not involved in the actual coding.  Here are some
names that come to mind.  Please send us mail if we have forgotten
any.  Tony Matranga, Tim Poplaski, Chris Adami, and everyone else who
participated in the FALL SBW and SPRING DOMINION games at Stony Brook
in the fall 1990 and spring 1991 semesters.

If you are interested in playing in any future games at Stony Brook,
please mail ``rosalia@max.physics.sunysb.edu.''

\end{document}
